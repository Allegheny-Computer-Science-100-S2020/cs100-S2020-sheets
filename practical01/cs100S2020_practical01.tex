\documentclass[11pt]{article}

% NOTE: The "Edit" sections are changed for each assignment

% Edit these commands for each assignment

\newcommand{\assignmentduedate}{January 23}
\newcommand{\assignmentassignedate}{January 17}
\newcommand{\assignmentnumber}{One}

\newcommand{\labyear}{2020}
\newcommand{\labdueday}{Thursday}
\newcommand{\labassignday}{Friday}
\newcommand{\labtime}{10:00 am}

\newcommand{\assigneddate}{Assigned: \labassignday, \assignmentassignedate, \labyear{} at \labtime{}}
\newcommand{\duedate}{Due: \labdueday, \assignmentduedate, \labyear{} at \labtime{}}

% Commands to describe key development tasks

% --> Running gatorgrader.sh
\newcommand{\gatorgraderstart}{\command{gradle grade}}
\newcommand{\gatorgradercheck}{\command{gradle grade}}

% --> Compiling and running and testing program with gradle
\newcommand{\gradlebuild}{\command{gradle build}}
\newcommand{\gradletest}{\command{gradle test}}
\newcommand{\gradlerun}{\command{gradle run}}

% Commands to describe key git tasks

% NOTE: Could be improved, problems due to nesting

\newcommand{\gitcommitfile}[1]{\command{git commit #1}}
\newcommand{\gitaddfile}[1]{\command{git add #1}}

\newcommand{\gitadd}{\command{git add}}
\newcommand{\gitcommit}{\command{git commit}}
\newcommand{\gitpush}{\command{git push}}
\newcommand{\gitpull}{\command{git pull}}

\newcommand{\gitcommitmainprogram}{\command{git commit src/main/java/practicalone/Swap.java -m "Your
descriptive commit message"}}

% Use this when displaying a new command

\newcommand{\command}[1]{``\lstinline{#1}''}
\newcommand{\program}[1]{\lstinline{#1}}
\newcommand{\url}[1]{\lstinline{#1}}
\newcommand{\channel}[1]{\lstinline{#1}}
\newcommand{\option}[1]{``{#1}''}
\newcommand{\step}[1]{``{#1}''}

\usepackage{pifont}
\newcommand{\checkmark}{\ding{51}}
\newcommand{\naughtmark}{\ding{55}}

\usepackage{listings}
\lstset{
  basicstyle=\small\ttfamily,
  columns=flexible,
  breaklines=true
}

\usepackage{fancyhdr}

\usepackage[margin=1in]{geometry}
\usepackage{fancyhdr}

\pagestyle{fancy}

\fancyhf{}
\rhead{Computer Science 100}
\lhead{Practical Assignment \assignmentnumber{}}
\rfoot{Page \thepage}
\lfoot{\duedate}

\usepackage{titlesec}
\titlespacing\section{0pt}{6pt plus 4pt minus 2pt}{4pt plus 2pt minus 2pt}

\newcommand{\labtitle}[1]
{
  \begin{center}
    \begin{center}
      \bf
      CMPSC 100\\Computational Expression\\
      Spring 2020\\
      \medskip
    \end{center}
    \bf
    #1
  \end{center}
}

\begin{document}

\thispagestyle{empty}

\labtitle{Practical \assignmentnumber{} \\ \assigneddate{} \\ \duedate{}}

% Slack for this course:

% https://join.slack.com/t/cs100spring2020/signup

\section*{Objectives}

To learn how to install and use a contained-based platform, called Docker, to
support the completion of technical activities (e.g., building and running a
Java program) during the class, laboratory, and practical sessions. In addition
to learning how to install and use a text editor such as Atom or VS Code,
students will also learn how to install a package manager such as Brew or
Chocolatey. After you install a package manager, you will use it to further
install programs like a terminal window and the Git command-line client. As
necessary, students may optionally configure their laptop's connection to a
GitHub account, although a later assignment will also explain the steps needed
to complete this task. Throughout this assignment, students will also learn how
to use Slack to support communication with each other, the student technical
leaders, and the instructor.

\section*{Suggestions for Success}

\begin{itemize}
  \setlength{\itemsep}{0pt}

\item {\bf Follow each step carefully}. Slowly read each sentence in the
  assignment sheet, making sure that you precisely follow each instruction. Take
  notes about each step that you attempt, recording your questions and ideas and
  the challenges that you faced. If you are stuck, then please tell a technical
  leader or instructor what assignment step you recently completed.

\item {\bf Regularly ask and answer questions}. Please log into Slack at the
  start of a laboratory or practical session and then join the appropriate
  channel. If you have a question about one of the steps in an assignment, then
  you can post it to the designated channel. Or, you can ask a student sitting
  next to you or talk with a technical leader or the course instructor.

% \item {\bf Store your files in GitHub}. Starting with this practical assignment,
%   you will be responsible for storing all of your files (e.g., Java source code
%   and Markdown-based writing) in a Git repository using GitHub Classroom. Please
%   verify that you have saved your source code in your Git repository by using
%   \command{git status} to ensure that everything is updated. You can see if your
%   assignment submission meets the established correctness requirements by using
%   the provided checking tools on your local computer and in checking the commits
%   in GitHub.

% \item {\bf Keep all of your files}. Don't delete your programs, output files,
%   and written reports after you submit them through GitHub; you will need them
%   again when you study for the quizzes and examinations and work on the other
%   laboratory, practical, and final project assignments.

% \item {\bf Back up your files regularly}. All of your files are regularly
%   backed-up to the servers in the Department of Computer Science and, if you
%   commit your files regularly, stored on GitHub. However, you may want to use a
%   flash drive, Google Drive, or your favorite backup method to keep an extra
%   copy of your files on reserve. In the event of any type of system failure, you
%   are responsible for ensuring that you have access to a recent backup copy of
%   all your files.

\item {\bf Explore teamwork and technologies}. While certain aspects of the
  course assignments will be challenging for you, each part is designed to give
  you the opportunity to learn both fundamental concepts in the field of
  computer science and explore advanced technologies that are commonly employed
  at a wide variety of companies. To explore and develop new ideas, you should
  regularly communicate with your team and/or the student technical leaders.

\item {\bf Hone your technical writing skills}. Computer science assignments
  require to you write technical documentation and descriptions of your
  experiences when completing each task. Take extra care to ensure that your
  writing is interesting and both grammatically and technically correct,
  remembering that computer scientists must effectively communicate and
  collaborate with their team members, the student technical leaders, and the
  course instructor.

\item {\bf Review the Honor Code on the syllabus}. While you may discuss your
  assignments with others, copying source code or writing is a violation of
  Allegheny College's Honor Code.

\end{itemize}

\section*{Reading Assignment}

If you have not done so already, please read all of the relevant ``GitHub
Guides'', available at \url{https://guides.github.com/}, that explain how to use
many of the features that GitHub provides. In particular, please make sure that
you have read guides such as ``Mastering Markdown'' and ``Documenting Your
Projects on GitHub''; each of them will help you to understand how to use both
GitHub and GitHub Classroom. In addition to reading all of the relevant
documentation about Docker that is available at \url{https://docs.docker.com/},
students should review the documentation about their chosen text editor (e.g.,
Vim, Atom, or VS Code) and package manager (e.g., Chocolatey or Brew); to
complete this part of the reading assignment, please visit the web site of each
tool.

After you have bookmarked references for each of the aforementioned web sites
and reviewed some of the printed guides, please access the Slack workspace that
we will use in this course, available for signup at
\url{https://join.slack.com/t/cs100spring2020/signup}. Once you have access to
the Slack workspace, please join the \url{#practicals} channel and later post
one thing that you have learned about Docker, your chosen text editor, and your
package manager. Along with posting what you learned, please review the
comments from other students, always attempting to post something that is not
already in the channel. Please make sure that you keep Slack open throughout
your completion of this practical assignment as you will continue to use it to
communicate with other students, the technical leaders, and the course
instructor, who will read and assess your insights.

\section*{Installing and Using A Text Editor}

You should first install a text editor, such as Atom or VS Code, which are
respectively available from \url{https://atom.io/} and
\url{https://code.visualstudio.com/}. Please follow the download instructions
for your operating system to install one or both of these text editors.
Optionally, you may also explore the installation of other editors such as Vim.
Students who want to install Vim or Neovim should talk with the course
instructor or a student technical leader. After choosing a text editor, return
to the \url{#practicals} channel and say which editor you picked and why you
selected it.

\section*{Installing and Using a Package Manager}

If you are using the Linux operating system, then you can use the default
\program{apt} program to install other programs. Since Windows and MacOS do not
provide a default package manager, you should install \program{choco} for
Windows and \program{brew} for MacOS. To learn more about these tools, you can
respectively visit \url{https://chocolatey.org/} and \url{https://brew.sh/}.
Please follow the installation instructions for these package managers, making
note of the challenges that you faced and the ways in which you solved them.
Finally, please share your experiences in the correct Slack channel.

After installing the package manager, you can use it to install other programs!
For example, the Windows operating system does not have a good terminal window
by default. To resolve this situation, you could install the new Windows Terminal
by using the Chocolatey commands at
\url{https://chocolatey.org/packages/microsoft-windows-terminal}. Since Linux
and MacOS have suitable terminal windows available by default, everyone can now
use a terminal to run package manager commands to install the \program{git}
command-line tool for their operating system. Before moving onto the next step
of the assignment, you should now have working versions of the following
software tools:

\begin{itemize}

  \item Text editor

  \item Package manager

  \item Terminal window

  \item Git command-line client

\end{itemize}

\section*{Installing and Using Docker}

Docker is a platform for software engineers and system administrators to
develop, deploy, and run software applications with containers that house
full-fledged executable programs. When software engineers use Docker, they no
longer have to focus on installing the programs that come inside of the Docker
container. Instead, they can use a program in a Docker container as if it was
installed on their development workstation, regardless of whether they are using
Windows, Linux, or MacOS. A wide variety of companies, such as Visa and PayPal,
use Docker. Throughout this semester, you will receive Docker containers that
house programs like Java, Gradle, and GatorGrader. These containers will also
hold the supporting libraries and packages needed to run the aforementioned
programs. With that said, please note that you should not store source code or
technical writing inside of a Docker container because it is a ``transient''
software application that does not store its state upon shut-down. Finally, the
provided Docker container will not include programs like Git or your text editor
as the instructor expects that each student will install these programs
separately.

The Department of Computer Science has adopted Docker for all computer science
courses. The course instructor, student technical leaders, and the departmental
systems administrator will support the use of Docker on laptops running
Windows, Linux, and MacOS. The instructor in this course will provide Docker
containers with all of the necessary software to run all class exercises and
laboratory and practical assignments. Students will then be able to develop and
run Java programs on their own laptops, using an instructor-provided Docker
container. Before being able to use Docker, students must first complete the
installation of it, following steps that are specific to the operating system
on your laptop. Bearing in mind that you must follow steps customized to your
laptop, please talk to the instructor or a technical leader if you get stuck on
these steps.

In advance of reading these installation instructions for Docker, you should
know that they are largely customized for students who use Windows or MacOS. If
you use the Linux operating system, then you can follow the installation
instructions available at this web site:
\url{https://docs.docker.com/install/linux/docker-ce/ubuntu/}. Okay, now you are
ready to install and use Docker!

\begin{figure}

\begin{verbatim}
docker run hello-world

docker : Unable to find image 'hello-world:latest' locally
...

latest:
Pulling from library/hello-world
ca4f61b1923c:
Pulling fs layer
ca4f61b1923c:
Download complete
ca4f61b1923c:
Pull complete
Digest: sha256:97ce6fa4b6cdc0790cda65fe7290b74cfebd9fa0c9b8c38e979330d547d22ce1
Status: Downloaded newer image for hello-world:latest

Hello from Docker!
This message shows that your installation appears to be working correctly.
...
\end{verbatim}

\vspace*{-.25in}
\caption{Example Output from Running a ``Hello World!''  Docker Container.}~\label{fig:docker}
\vspace*{-.25in}
\end{figure}

% For older MacOS and Windows operating systems, including Windows 10 Home, please
% follow the installation instructions for Docker Toolbox, given at
% \url{https://docs.docker.com/toolbox/toolbox_install_windows/}.

\begin{enumerate}

    \item Fist, determine the version of your operating system. If your laptop
      meets the requirements outlined below, then you can proceed to the next
      step. Please note that the Department of Computer Science at Allegheny
      College exclusively supports the use of Docker on Windows 10 Pro, Windows
      10 Enterprise, MacOS, and Linux. All other operating systems (e.g., Chrome
      OS, Windows 7, Windows 8, or Windows 10 Home) are not officially supported
      and may not work reliably with legacy programs such as Docker Toolbox.
      Students who are using an unsupported version of Windows (e.g., Windows 10
      Home), should purchase an upgrade to Windows 10 Pro or use one of the
      department-provided laptops or desktops. In summary, here are the
      specifications required to run the Docker Desktop program used in this
      course:

      \textbf{Docker Desktop Specifications} \\
      -- Mac: \vspace{-.1in}
      \begin{itemize}
        \setlength{\itemsep}{0in}
        \item 2010 model or newer with hardware support for MMU, EPT, and Unrestricted Mode
        \item MacOS Sierra 10.12 or newer
        \item 4 GB of RAM
        \item VirtualBox prior to version 4.3.30 cannot be installed
      \end{itemize}
      -- Windows: \vspace{-.1in}
      \begin{itemize}
        \item Windows 10 64--bit Pro, Enterprise, or Education (Build 15063 or later)
        \item Virtualization is enabled in BIOS
        \item CPU SLAT-capable feature
        \item 4 GB of RAM
      \end{itemize}
      -- Linux: \vspace{-.1in}
      \begin{itemize}
        \item 64--bit
        \item Kernel 3.10 or later
      \end{itemize}

    \item Go to \url{https://docs.docker.com/install/} and, from the menu on
      the left-hand side, select and follow the installation tutorial for your
      operating system (Linux, MacOS, or Windows). For example, if you have
      MacOS you can find the installation guide at
      \url{https://docs.docker.com/docker-for-mac/install/}. If you have the
      Windows operating system, please follow the installation guide at
      \url{https://docs.docker.com/docker-for-windows/install/}. Students with
      Linux can use a package manager when following the instructions at the
      aforementioned site.

      Please note that the Docker Desktop for Mac and Windows requires you to
      create an account to download the platform. To bypass this, you can use
      the following direct download links:

    \begin{itemize}
        \item Windows: \\ \url{https://download.docker.com/win/stable/Docker\%20for\%20Windows\%20Installer.exe}
        \item Mac: \url{https://download.docker.com/mac/stable/Docker.dmg}
    \end{itemize}

  \item Once your Docker installation is complete, please run a terminal window
    on your machine. In MacOS and Linux, you can search for a ``terminal'' and
    on Windows machines look for \command{cmd}. If you correctly followed the
    previous assignment steps and you use Windows, then you can also use the new
    ``terminal'' that you installed. Now, in the terminal window you can type
    \command{docker run hello-world} command and press ``Enter''. If this step
    does not work, please read the troubleshooting section at the end of the
    sheet and work with the student technical leaders and the course instructor
    to resolve the problem. If you see an output similar to the one in
    Figure~\ref{fig:docker} you can high five a technical leader as your Docker
    set up was successful! If you got this step to work, please  help the other
    students at your table who are stuck on this step and reflect on the meaning
    of the output you see in your terminal window.

  \item Note that even though the previous Docker container caused your terminal
    to print a diagnostic message, it did not allow you to run any programs.
    With that said, it is also possible to download and use a Docker container
    that contains the complete version of an operating system! To explore this
    feature of Docker, please type the command \command{docker run -it ubuntu
    /bin/bash} and notice that it produces output like that found in
    Figure~\ref{fig:ubuntu}. Congratulations, you are now running the Linux
    operating system inside a container running on your host operating system!
    If you got stuck on this step, please ask the course instructor for help.

  \item You can use commands line \command{cd} and \command{ls} to explore the
    default configuration of this Ubuntu container provided by Docker. Outside
    of Docker, type the commands \command{docker ps} and \command{docker images}
    and explain this output to a technical leader and a fellow student. You
    should also paste the output of these commands into the \url{#practicals}
    channel in Slack. Remember, you can paste a block of formatted text into
    Slack by using three ``backticks'', pasting the content, and then closing
    with three additional backticks. Finally, please write in Slack about what
    you experienced during the installation of Docker. What did you find
    confusing? Why did you find it to be confusing? Based on your experiences in
    this practical assignment, what do you think are the benefits associated
    with using Docker in a computer science course?

\end{enumerate}


\begin{figure}

\begin{verbatim}
Unable to find image `ubuntu:latest' locally
latest: Pulling from library/ubuntu
35c102085707: Pull complete
251f5509d51d: Pull complete
8e829fe70a46: Pull complete
6001e1789921: Pull complete
Digest: sha256:d1d454df0f579c6be4d8161d227462d69e163a8ff9d20a847533989cf0c94d90
Status: Downloaded newer image for ubuntu:latest
\end{verbatim}

\vspace*{-.25in}
\caption{Example Output from Running a Ubuntu Docker Container.}~\label{fig:ubuntu}
\vspace*{-.25in}
\end{figure}

\vspace*{-.1in}

After completing each stage of this practical assignment, you should have these
correctly working versions of the following programs installed on your laptop,
regardless of your operating system.

\vspace*{-.05in}

\begin{itemize}
  \setlength{\itemsep}{0pt}

  \item Text editor

  \item Package manager

  \item Terminal window

  \item Git command-line client

  \item Docker container platform with:

    \vspace*{-.1in}
    \begin{itemize}
      \setlength{\itemsep}{0pt}

      \item Docker ``Hello World!'' container
      \item Docker Ubuntu container

    \end{itemize}

\end{itemize}

\vspace*{-.1in}

Every student in this course is must have access to a Windows, MacOS, or Linux
laptop that correctly runs these programs, in full support of forthcoming Java
software development projects.

\section*{Troubleshooting for Common Challenges}

Installing and using Docker Desktop can be difficult and common for you may have
to try this step multiple times to overcome the following technical challenges.

\begin{itemize}

\setlength{\itemsep}{0in}

  \item {\bf Incorrect Operating System}: If Docker does not work correctly,
    please make sure that you are using one of the officially supported versions
    of Windows, MacOS, or Linux.

  \item {\bf Out-of-date Packages}: If you are running the correct version of an
    operating system, you may still have to update some of the libraries,
    packages, and programs that are installed on your laptop in order to ensure
    that all programs work correctly.

  \item {\bf Incorrect File System Permissions}: Since Docker needs access to
    the file system on your laptop's hard disk, you should ensure that you have
    shared your central disk with Docker. For instance, on Windows 10 Pro this
    normally requires you to share the \command{C:\\} drive.

  \item {\bf Insufficient Hardware Resources}: If Docker crashes after starting
    or produces error messages about limited memory, it may not be possible to
    use Docker until you close all other programs on your laptop. Docker may not
    run well or at all on resource-constrained laptops.

  \item {\bf Incorrect Privileges}: If Docker or your package manager will not
    run correctly, it is possible that you are starting it in a fashion that
    does not afford it ``administrator privileges'' and thus it cannot access
    your file system or operating system. If this happens on Windows 10 Pro, you
    may need to ensure that you run your terminal window as an administrator.

\end{itemize}

\vspace*{-.1in}

\section*{Technical Questions to Answer in Slack}

\noindent Along with providing all the required program outputs, please answer
these questions in Slack.

\vspace*{-.05in}

\begin{itemize}

\setlength{\itemsep}{0in}

  \item What is a package manager and why will you use it in this course?

  \item What is a terminal window and how will you use it during this course?

  \item What is a text editor and why will you use it instead of Microsoft Word?

  \item What is Docker and what are the benefits of using it in this course?

  \item What is your most pressing concern about the Computer Science 100 course?

\end{itemize}

\vspace*{-.1in}

\section*{Summary of the Required Deliverables}

% \noindent Students do not need to submit printed source code or technical
% writing for any assignment in this course.

\noindent This assignment invites you to submit, using \url{#practicals} in
Slack, the following deliverables:

\vspace*{-.025in}

\begin{enumerate}

  \setlength{\itemsep}{0in}

  \item A description of the challenges that you experienced and the solutions
    that you developed for each challenge, following each of the prompts
    previously given in the assignment sheet.

  \item Along with answer the above questions, the correctly formatted and
    explained output from running Docker commands like \command{docker ps} and
    \command{docker images} in your terminal window.

\end{enumerate}

\vspace*{-.1in}

\section*{Evaluation of Your Practical Assignment}

All practical assignments in this course are graded on a mastery, or
credit-no-credit, basis. For this assignment, you will receive credit as long as
you type all of the required information into the \url{#practicals} channel in
Slack. You will know that you earned the checkmark grade for this assignment if
you can see a checkmark emoji next to a comment that you made in the appropriate
channel. Please bear in mind that, for future practical and laboratory
assignments, you will submit all of your source code and technical writing
through a GitHub repository created by GitHub Classroom. However, for this
assignment, you must instead submit all of your work through Slack.

\section*{Adhering to the Honor Code}

In adherence to the Honor Code, students should complete this assignment on an
individual basis. While it is appropriate for students in this class to have
high-level conversations about the assignment, it is necessary to distinguish
carefully between the student who discusses the principles underlying a problem
with others and the student who produces assignments that are identical to, or
merely variations on, someone else's work. Deliverables (e.g., Java source code
or Markdown-based technical writing) that are nearly identical to the work of
others will be taken as evidence of violating the \mbox{Honor Code}. Please see
the course instructor if you have questions about this policy.


\end{document}
