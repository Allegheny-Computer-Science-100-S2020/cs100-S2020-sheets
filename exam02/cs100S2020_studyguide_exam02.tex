\documentclass[11pt]{article}

% NOTE: The "Edit" sections are changed for each assignment

% Edit these commands for each assignment

\newcommand{\assignmentduedate}{December 13}
\newcommand{\assignmentassignedate}{December 9}
\newcommand{\assignmentnumber}{Two}

\newcommand{\labyear}{2019}
\newcommand{\assignedday}{Monday}
\newcommand{\dueday}{Friday}
\newcommand{\labtime}{7:00 pm}

\newcommand{\assigneddate}{Announced: \assignedday, \assignmentassignedate, \labyear{}}
\newcommand{\duedate}{Exam: \dueday, \assignmentduedate, \labyear{} at \labtime{}}

% Edit these commands to give the name to the main program

\newcommand{\mainprogram}{\lstinline{DisplayOutput}}
\newcommand{\mainprogramsource}{\lstinline{src/main/java/labone/DisplayOutput.java}}

% Edit this commands to describe key deliverables

\newcommand{\reflection}{\lstinline{writing/reflection.md}}

% Commands to describe key development tasks

% --> Running gatorgrader.sh
\newcommand{\gatorgraderstart}{\command{./gatorgrader.sh --start}}
\newcommand{\gatorgradercheck}{\command{./gatorgrader.sh --check}}

% --> Compiling and running program with gradle
\newcommand{\gradlebuild}{\command{gradle build}}
\newcommand{\gradlerun}{\command{gradle run}}

% Commands to describe key git tasks

% NOTE: Could be improved, problems due to nesting

\newcommand{\gitcommitfile}[1]{\command{git commit #1}}
\newcommand{\gitaddfile}[1]{\command{git add #1}}

\newcommand{\gitadd}{\command{git add}}
\newcommand{\gitcommit}{\command{git commit}}
\newcommand{\gitpush}{\command{git push}}
\newcommand{\gitpull}{\command{git pull}}

\newcommand{\gitcommitmainprogram}{\command{git commit src/main/java/labone/DisplayOutput.java -m "Your
descriptive commit message"}}

% Use this when displaying a new command

\newcommand{\command}[1]{``\lstinline{#1}''}
\newcommand{\program}[1]{\lstinline{#1}}
\newcommand{\url}[1]{\lstinline{#1}}
\newcommand{\channel}[1]{\lstinline{#1}}
\newcommand{\option}[1]{``{#1}''}
\newcommand{\step}[1]{``{#1}''}

\usepackage{pifont}
\newcommand{\checkmark}{\ding{51}}
\newcommand{\naughtmark}{\ding{55}}

\usepackage{listings}
\lstset{
  basicstyle=\small\ttfamily,
  columns=flexible,
  breaklines=true
}

\usepackage{fancyhdr}

\usepackage[margin=1in]{geometry}
\usepackage{fancyhdr}

\pagestyle{fancy}

\fancyhf{}
\rhead{Computer Science 100}
\lhead{Exam \assignmentnumber{}}
\rfoot{Page \thepage}
\lfoot{\duedate}

\usepackage{titlesec}
\titlespacing\section{0pt}{6pt plus 4pt minus 2pt}{4pt plus 2pt minus 2pt}

\newcommand{\guidetitle}[1]
{
  \begin{center}
    \begin{center}
      \bf
      CMPSC 100\\Computational Expression\\
      Fall 2019\\
      \medskip
    \end{center}
    \bf
    #1
  \end{center}
}

\begin{document}

\guidetitle{Exam \assignmentnumber{} Study Guide \\ \assigneddate{} \\ \duedate{}}

\section*{Introduction}

This course will have its final exam on Friday, \assignmentduedate{}, 2017 from
7:00 pm to 10:00 pm. The exam will be ``closed notes'' and ``closed book'' and
it will cover the following materials. Please review the ``Course Schedule'' on
the Web site for the course to see the content and slides that we have covered
to this date. Students may post questions about this material to our Slack
workspace. The instructor encourages students to form study groups to review for
this upcoming examination.

\begin{itemize}

  \itemsep 0in

  \item Chapter One in Lewis \& Loftus (i.e., ``Introduction to Computation and Programming'')

  \item Chapter Two in Lewis \& Loftus, Sections 2.1--2.9 (i.e., ``Data and Expressions'')

  \item Chapter Three in Lewis \& Loftus, Sections 3.1--3.7 (i.e., ``Using Classes and Objects'')

  \item Chapter Four in Lewis \& Loftus, Sections 4.1--4.9 (i.e., ``Writing Classes'')

  \item Chapter Five in Lewis \& Loftus, Sections 5.1--5.6 (i.e., ``Conditionals and Loops'')

  \item Chapter Six in Lewis \& Loftus, Sections 6.1--6.4 (i.e., ``More Conditionals and Loops'')

  \item Chapter Eight in Lewis \& Loftus, Sections 8.1--8.4 (i.e., ``Arrays'')

  \item Chapter Eleven in Lewis \& Loftus, Sections 11.1--11.6 (i.e, ``Exceptions'')

  \item Chapter Twelve in Lewis \& Loftus, Sections 12.1--12.3 (i.e., ``Recursion'')

  \item Using the many commands on your laptop's operating system; source code
    editing, compiling and executing programs in Docker; knowledge of the basic
    commands for using GitHub.

  % NOTE: Did two fewer practical assignments due to GitHub problems and a
  % hurricane warning in Meadville, and a power outage

  % NOTE: Did fewer labs and practicals again because of scheduling and class
  % constraints

  \item Your class notes and lecture slides, labs 1--9, practicals 1--9, and the
    handouts from lab.

\end{itemize}

\noindent Like the past quiz and exam, this exam will be a mix of questions that
have a form such as fill in the blank, short answer, true/false, and completion.
The emphasis will be on the following topics:

\vspace*{-.05in}
\begin{itemize}

  \itemsep 0in

  \item Fundamental concepts in computing and the Java language (e.g., definitions and background).

  \item Fundamental concepts in programming languages (e.g., conditional logic and iteration).

  \item Advanced concepts in programming languages (e.g., exception handling and recursion).

  \item Practical laboratory techniques (e.g., editing, compiling, and running programs; effectively using files and
    directories; correctly using GitHub through the command-line {\tt git} program).

  \item Understanding Java programs (e.g., given a short, perhaps even one line, source code segment written in Java,
    understand what it does and be able to precisely describe its output).

  \item Composing Java statements and programs, given a description of what should be done. Students should be completely
    comfortable writing short source code statements that are in nearly correct form as Java code. While your program may
    contain small syntactic errors, it is not acceptable to ``make up'' features of the Java programming language that do
    not exist in the language itself---so, please do not call a ``{\tt solveQuestionThree()}'' method!

\end{itemize}

\noindent No partial credit will be given for questions that are true/false,
completion, or fill in the blank. Minimal partial credit may be awarded for the
questions that require a student to write a short answer. You are strongly
encouraged to write short, precise, and correct responses to all of the
questions. When you are taking the exam, you should do so as a ``point
maximizer'' who first responds to the questions that you are most likely to
answer correctly for full points. Please make sure that you review the past
assessments so that you can comfortably answer their questions. Students should
keep the time limitation in mind as they are absolutely required to submit the
examination at the end of the period unless they have written permission for
extra time from a member of the Learning Commons. Students who do not submit
their exam on time will have their overall point total reduced. Please see the
instructor if you have questions about these policies.

\section*{Reminder Concerning the Honor Code}

\noindent Students are required to fully adhere to the Honor Code during the
completion of this exam. More details about the Allegheny College Honor Code are
provided on the syllabus. Students are strongly encouraged to carefully review
the full statement of the Honor Code before taking this exam. The following
provides you with a review of Honor Code statement from the course syllabus:

The Academic Honor Program that governs the entire academic program at Allegheny
College is described in the Allegheny Academic Bulletin. The Honor Program
applies to all work that is submitted for academic credit or to meet non-credit
requirements for graduation at Allegheny College. This includes all work
assigned for this class (e.g., examinations, laboratory assignments, and the
final project). All students who have enrolled in the College will work under
the Honor Program. Each student who has matriculated at the College has
acknowledged the following pledge:

\begin{quote}
%
  I hereby recognize and pledge to fulfill my responsibilities, as defined in
  the Honor Code, and to maintain the integrity of both myself and the College
  community as a whole.
%
\end{quote}

Students who have questions about Allegheny College's Honor Code and how it
applies to the completion of a quiz or an examination in Computer Science 100,
should immediately schedule a meeting with the course instructor to openly
discuss their questions and concerns.

\section*{Detailed Review of Content}

The listing of topics in the following subsections is not exhaustive; rather, it serves to illustrate the types of
concepts that students should study as they prepare for the exam. Please see the instructor during office hours if you
have questions about any of the content listed in this section.

\subsection*{Chapter One}

\begin{itemize}

  \itemsep 0in

  \item Basic understanding of computer hardware and software
  \item Knowledge of computer number systems (e.g., binary and decimal)
  \item Purpose for and steps of the fetch-decode-execute cycle in the CPU
  \item Layout of and access techniques for computer memory
  \item Knowledge of computer networking methods and programs
  \item Basic syntax and semantics of the Java programming language
  \item Fundamental principles of the object-oriented programming paradigm
  \item Appropriate ways to write and add comments to a Java program
  \item Input(s) and output(s) of the Java compiler and virtual machine
  \item Input(s) and output(s) of the Java compiler and virtual machine
  \item How to use \command{docker} to enter and use a software container
  \item How to use \command{gradle} to build and run a Java program

\end{itemize}

\subsection*{Chapter Two}

\begin{itemize}

  \itemsep 0in

  \item Using escape sequences to control the output of Java programs

  \item How to import existing Java classes into a Java program

  \item Ways to perform input and output in a Java program

  \item An understanding of the ``fully qualified name'' of a Java class

  \item Similarities and differences between file and console input and output

  \item The variety of data types available for storing numerical and character
    data

  \item The trade-offs associated with using different data types (e.g.,
    \program{int} and \program{float})

  \item The similarities and differences between ``primitives'' and
    ``references''

  \item The declaration of variables and assignment of values to variables

  \item Operators and operator precedence in Java expressions

  \item Techniques for converting variables from one data type to another

\end{itemize}

\subsection*{Chapter Three}

\begin{itemize}

  \itemsep 0in

  \item The steps for creating a new instance of a Java class
  \item How to use technical diagrams to visualize an object in memory
  \item The meaning of the term ``alias'' for the variables in a Java program
  \item The creation and use of Strings in a Java program (e.g., substring operations)
  \item The ways in which Java packages promote the effective organization of a program
  \item The variety of ways in which you can create and use random numbers
  \item How to call and use the methods provided by the {\tt java.lang.Math} class
  \item The ways in which programs can create formatted output in a terminal window
  \item Computer graphics and related topics such as pixels and screen resolution
  \item The coordinate system used to display computer graphics in Java programs
  \item The use of the RGB system for specifying colors in Java programs
  \item The ways in which Java programs can modify the pixels in an image
  % \item How to call Java methods to display colored shapes on the screen
  % \item How to use arithmetic expressions to calculate colors and layouts

\end{itemize}

\subsection*{Chapter Four}

\begin{itemize}

  \itemsep 0in

  \item Best practices for organizing and structuring the variables and methods
    of a Java class

  \item How to use a technical diagram to describe the design of a Java program

  \item How to declare an instance variable and an instance method in a Java
    class

  \item The ways in which encapsulation promotes good object-oriented program
    design

  \item The visibility modifiers (e.g., {\tt private}) that support
    encapsulation in the Java language

  \item The flow of control following the invocation of multiple methods in a
    Java program

  \item The meaning and purpose of accessor and mutator methods in Java programs

  \item All of the key parts of a Java method (e.g., parameters and {\tt
    return} statements)

  \item The meaning of the method-related terms ``formal parameter'' and
    ``actual parameter''

  \item How the Java programming language supports parameter passing between
    methods

  \item Appropriate strategies for implementing constructors in a Java class

  \item The similarities and differences between constructors and
    general-purpose methods

  \item A basic understanding of the graphical objects seen in user interfaces

\end{itemize}

\subsection*{Chapter Five}

\begin{itemize}

  \itemsep 0in

  \item The meaning and purpose of Boolean expressions in conditional logic

  \item The different logical operators available for use in Boolean expressions

  \item The equality and relational operators available for use in Boolean expressions

  \item The overall structure and purpose of {\tt if} statements in the Java
    language

  \item How to correctly implement and format conditional logic statements in
    Java

  \item How to use a truth table to understand the meaning of {\tt if}
    statements in Java

  \item The challenges associated with comparing variables of different data
    types

  \item Challenges of and best practices for comparing variables of different
    data types

  \item The meaning and purpose of looping constructs in the Java language

  \item The overall structure and purpose of {\tt while} statements in the Java
    language

  \item An intuitive understanding of the challenges associated with infinite
    loops

  \item How {\tt break} and {\tt continue} statements change the behavior of looping constructs

  \item How to use an {\tt Iterator} to loop through all of the values in an
    {\tt ArrayList}

  \item How to draw technical diagrams that explain the control flow of a Java
    program

\end{itemize}

\subsection*{Chapter Six}

\begin{itemize}

  \itemsep 0in

  \item The meaning and purpose of {\tt switch} statements in the Java language
  \item The purpose of the {\tt default} case in a {\tt switch} statement
  \item The overall structure and purpose of {\tt do-while} statements in the Java language
  \item How {\tt do-while} statements are similar to and different from {\tt while} statements
  \item How to draw a technical diagram to represent the behavior of a looping construct
  \item The meaning and purpose of {\tt for} loops in the Java programming language
  \item How to use {\tt for} loops to compute and display multiples of a number
  \item How to use (potentially nested) {\tt for} loops to display geometric shapes in the terminal window
  \item Best practices for picking a specific looping construct for a problem requiring iteration
  \item The ability to study a looping construct and determine how many times it will execute
  \item The ability to study a looping construct and determine if it will loop
    infinitely

\end{itemize}

\subsection*{Chapter Eight}

\begin{itemize}

  \itemsep 0in

  \item An example of a problem that is best solved through the use of an array
  \item The types of technical diagrams that are best suited to visualizing an array
  \item The benefits and drawbacks associated with using arrays in a Java program
  \item The means by which you declare and use arrays in the Java programming language
  \item The ways in which arrays can store either primitive variables or objects
  \item The key characteristics of the array data structure (e.g., stores a single type of data)
  \item The meaning of the word ``index'' and how it connects to the array data structure
  \item The meaning and purpose of arrays bounds checking in the Java programming language
  \item The benefits and drawbacks associated with array bounds
    checking during program execution
  \item How to combine {\tt for} loops and arrays in order to solve a computational problem
  \item How arrays are used to accept command-line arguments as input to a program
  \item An intuitive understanding of the meaning and purpose of two-dimensional arrays
  \item Understanding of the similarities and differences between one- and two-dimensional arrays
  \item Knowledge of how arrays and {\tt ArrayList}s are similar to and different from each other

\end{itemize}

\end{document}
