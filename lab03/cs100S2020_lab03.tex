\documentclass[11pt]{article}

% NOTE: The "Edit" sections are changed for each assignment

% Edit these commands for each assignment

\newcommand{\assignmentduedate}{February 13}
\newcommand{\assignmentassignedate}{February 6}
\newcommand{\assignmentnumber}{Three}

\newcommand{\labyear}{2020}
\newcommand{\labday}{Thursday}
\newcommand{\labtime}{2:30 pm}

\newcommand{\assigneddate}{Assigned: \labday, \assignmentassignedate, \labyear{} at \labtime{}}
\newcommand{\duedate}{Due: \labday, \assignmentduedate, \labyear{} at \labtime{}}

% Edit these commands to give the name to the main program

\newcommand{\mainprogram}{\lstinline{TipCalculator}}
\newcommand{\mainprogramsource}{\lstinline{src/main/java/labthree/TipCalculator.java}}
\newcommand{\mainprograminput}{\lstinline{input/tip_inputs.txt}}

% Edit this commands to describe key deliverables

\newcommand{\reflection}{\lstinline{writing/reflection.md}}

% Commands to describe key development tasks

% --> Running gatorgrader
\newcommand{\gatorgraderstart}{\command{gradle grade}}
\newcommand{\gatorgradercheck}{\command{gradle grade}}

% --> Compiling and running program with gradle
\newcommand{\gradlebuild}{\command{gradle build}}
\newcommand{\gradlerun}{\command{gradle run}}

% Commands to describe key git tasks

\newcommand{\gitcommitfile}[1]{\command{git commit #1}}
\newcommand{\gitaddfile}[1]{\command{git add #1}}

\newcommand{\gitadd}{\command{git add}}
\newcommand{\gitcommit}{\command{git commit}}
\newcommand{\gitpush}{\command{git push}}
\newcommand{\gitpull}{\command{git pull}}

\newcommand{\gitaddmainprogram}{\command{git add src/main/java/labthree/TipCalculator.java}}
\newcommand{\gitcommitmainprogram}{\command{git commit src/main/java/labthree/TipCalculator.java -m "Your
descriptive commit message"}}

% Use this when displaying a new command

\newcommand{\command}[1]{``\lstinline{#1}''}
\newcommand{\program}[1]{\lstinline{#1}}
\newcommand{\url}[1]{\lstinline{#1}}
\newcommand{\channel}[1]{\lstinline{#1}}
\newcommand{\option}[1]{``{#1}''}
\newcommand{\step}[1]{``{#1}''}

% Enable margin notes to catch student attention

\usepackage{marginnote}
\reversemarginpar
\renewcommand*{\raggedrightmarginnote}{\centering}

% Load extra fonts

\usepackage{pifont}
\newcommand{\checkmark}{\ding{51}}
\newcommand{\naughtmark}{\ding{55}}
\usepackage{fontawesome}

% Commands for margin notes, for optional use

\newcommand{\book}{\faBook}
\newcommand{\group}{\faUsers}

% Commands for making key margin notes, fully integrated

\newcommand{\caution}[1]{\null\hfill\LARGE{\faWarning{}}\newline\scriptsize{\em{#1}}}
\newcommand{\discuss}[1]{\null\hfill\LARGE{\faCommentO{}}\newline\scriptsize{\em{#1}}}
\newcommand{\resource}[1]{\null\hfill\LARGE{\faLink{}}\newline\scriptsize{\em{#1}}}
\newcommand{\think}[1]{\null\hfill\LARGE{\faCogs{}}\newline\scriptsize{\em{#1}}}

\usepackage{listings}
\lstset{
  basicstyle=\small\ttfamily,
  columns=flexible,
  breaklines=true
}

\usepackage{fancyhdr}

\usepackage[margin=1in]{geometry}
\usepackage{fancyhdr}

\pagestyle{fancy}

\fancyhf{}
\rhead{Computer Science 100}
\lhead{Laboratory Assignment \assignmentnumber{}}
\rfoot{Page \thepage}
\lfoot{\duedate}

\usepackage{titlesec}
\titlespacing\section{0pt}{6pt plus 4pt minus 2pt}{4pt plus 2pt minus 2pt}

\newcommand{\labtitle}[1]
{
  \begin{center}
    \begin{center}
      \bf
      CMPSC 100\\Computational Expression\\
      Fall 2020\\
      \medskip
    \end{center}
    \bf
    #1
  \end{center}
}

\begin{document}

\thispagestyle{empty}

\labtitle{Laboratory \assignmentnumber{} \\ \assigneddate{} \\ \duedate{}}

\section*{Objectives}

To continue to practice using GitHub to access the files for a laboratory
assignment. Additionally, to master the use of the your operating system and
software development programs such as a ``terminal window'' and the ``Docker
desktop''. You will also continue to practice using Slack to support
communication with the teaching assistants and the course instructor. Next, you
will learn how to implement a Java program that reads input values from a file
and displays output to the user. Finally, you will continue to explore the use
of variables with different data types.

\section*{Suggestions for Success}

\begin{itemize}
  \setlength{\itemsep}{0pt}

\item {\bf Follow each step carefully}. Slowly read each sentence in the
  assignment sheet, making sure that you precisely follow each instruction. Take
  notes about each step that you attempt, recording your questions and ideas and
  the challenges that you faced. If you are stuck, then please tell a technical
  leader or instructor what assignment step you recently completed.

\item {\bf Regularly ask and answer questions}. Please log into Slack at the
  start of a laboratory or practical session and then join the appropriate
  channel. If you have a question about one of the steps in an assignment, then
  you can post it to the designated channel. Or, you can ask a student sitting
  next to you or talk with a technical leader or the course instructor.

\item {\bf Store your files in GitHub}. Starting with this laboratory
  assignment, you will be responsible for storing all of your files (e.g., Java
  source code and Markdown-based writing) in a Git repository using GitHub
  Classroom. Please verify that you have saved your source code in your Git
  repository by using \command{git status} to ensure that everything is
  updated. You can see if your assignment submission meets the established
  correctness requirements by using the provided checking tools on your local
  computer and by checking the commits in GitHub.

\item {\bf Keep all of your files}. Don't delete your programs, output files,
  and written reports after you submit them through GitHub; you will need them
  again when you study for the quizzes and examinations and work on the other
  laboratory, practical, and final project assignments.

\item {\bf Explore teamwork and technologies}. While certain aspects of the
  laboratory assignments will be challenging for you, each part is designed to
  give you the opportunity to learn both fundamental concepts in the field of
  computer science and explore advanced technologies that are commonly employed
  at a wide variety of companies. To explore and develop new ideas, you should
  regularly communicate with your team and/or the student technical leaders.

\item {\bf Hone your technical writing skills}. Computer science assignments
  require to you write technical documentation and descriptions of your
  experiences when completing each task. Take extra care to ensure that your
  writing is interesting and both grammatically and technically correct,
  remembering that computer scientists must effectively communicate and
  collaborate with their team members and the student technical leaders and
  course instructor.

\item {\bf Review the Honor Code on the syllabus}. While you may discuss your
  assignments with others, copying source code or writing is a violation of
  Allegheny College's Honor Code.

\end{itemize}

\section*{Reading Assignment}

If you have not done so already, please read all of the relevant ``GitHub
Guides'', available at \url{https://guides.github.com/}, that explain how to use
many of the features that GitHub provides. In particular, please make sure that
you have read guides such as ``Mastering Markdown'' and ``Documenting Your
Projects on GitHub''; each of them will help you to understand how to use both
GitHub and GitHub Classroom. To do well on this assignment, you should also
review Chapter 1 and study Chapter 2 in the course textbook, paying close
attention to Sections 2.1 through 2.6, Figure 2.7, and Listing 2.8. See the
instructor if you have questions these reading assignments.

\section*{Accessing the Laboratory Assignment on GitHub}

To access the laboratory assignment, you should go into the
\channel{\#announcements} channel in our Slack team and find the announcement
that provides a link for it. Copy this link and paste it into your web browser.
Now, you should accept the laboratory assignment and see that GitHub Classroom
created a new GitHub repository for you to access the assignment's starting
materials and to store the completed version of your assignment. Specifically,
to access your new GitHub repository for this assignment, please click the green
``Accept'' button and then click the link that is prefaced with the label ``Your
assignment has been created here''. If you accepted the assignment and correctly
followed these steps, you should have created a GitHub repository with a name
like
``Allegheny-Computer-Science-100-Fall-2020/computer-science-100-fall-2020-lab-3-gkapfham''.
Unless you provide the instructor with documentation of the extenuating
circumstances that you are facing, not accepting the assignment means that you
automatically receive a failing grade for it.

Before you move to the next step of this assignment, please make sure that you
read all of the content on the web site for your new GitHub repository, paying
close attention to the technical details about the commands that you will type
and the output that your program must produce. Now you are ready to download the
starting materials to your laboratory computer. Click the ``Clone or download''
button and, after ensuring that you have selected ``Clone with SSH'', please
copy this command to your clipboard. To enter into the right directory you
should now type \command{cd cs100F2020}. Next, you can type the command
\command{ls} and see that there are some files or directories inside of this
directory. By typing \command{git clone} in your terminal and then pasting in
the string that you copied from the GitHub site you will download all of the
code for this assignment. For instance, if the course instructor ran the
\command{git clone} command in the terminal, it would look like:

\begin{lstlisting}
  git clone git@github.com:Allegheny-Computer-Science-100-F2020/computer-science-100-fall-2020-lab-3-gkapfham.git
\end{lstlisting}

After this command finishes, you can use \command{cd} to change into the new
directory. If you want to \step{go back} one directory from your current
location, then you can type the command \command{cd ..}. Please continue to use
the \command{cd} and \command{ls} commands to explore the files that you
automatically downloaded from GitHub. What files and directories do you see?
What do you think is their purpose? Spend some time exploring, sharing your
discoveries with a neighbor and a \mbox{student technical leader}.

\section*{Reading from a File and Computing with Variables}

Today, you will view and improve another Java program in your text editor! To
get started, make sure that you can load the Java source code in your text
editor and find the Markdown-based file in which you will include your technical
writing. After you have started a terminal window and entered into the provided
Docker container, build the program and notice that it is not compiling
correctly! After you have solved this issue, please make sure that you can
explain the meaning and purpose of the following Java source code segments. If
you are not sure how these code segments influence the compilation and execution
of the program, please talk with the course instructor.

\begin{itemize}

  \item \program{import java.io.FileNotFoundException;}
  \item \program{import java.util.Scanner;}
  \item \program{String name = scanner.nextLine();}
  \item \program{System.out.println(name + ", welcome to the tip calculator.");}

\end{itemize}

\marginnote{\think{Research file input}}[.5in] This assignment invites you to
implement a Java program that displays exactly ten lines of output. The first
output line should display your name and then today's date. You can accomplish
the first task by typing \command{System.out.println("Gregory M. Kapfhammer " +
new Date());}, substituting your name for the name of the course instructor.
Your program will produce nine additional lines of output once you add more
\command{System.out.println} statements. Unlike your previous Java programs,
this Java program must read input values from a file and then use these inputs
to perform computations and to produce outputs. Please see the Java source code
comments for more details! The remainder of this section summarizes the purpose
of this program for tip calculation.

This assignment asks you to write a Java program, \mainprogramsource{}, that
will calculate the tip, the total bill for the user, and each person's share of
the restaurant bill (if there are two or more people). In particular, your
program needs to do the following:

\vspace*{-.1in}
\begin{enumerate}

  \item Ask the user to enter a name (remember to save the user's input into a
    variable so that it can be used later in the program and in the program's
    output).

  \item Using this name, display a friendly and appropriate welcome message to
    the user.

  \item Ask the user to enter the restaurant's bill amount (remember to save the
    user's input into an appropriate variable). You should allow for floating
    point (e.g., decimal or fractional) values.

  \item Ask the user to enter the desired tip percentage as a number between $0$
    and $100$ or as a decimal number between $0$ and $1$. You have to decide
    which range you want to use for your program and specify it when prompting
    for the user's input (after picking the correct data type for the
    percentage, remember to save the user's input into an appropriate variable).

  \item Calculate the tip as $tip = \frac{percentage}{100} \times bill$ if the
    percentage ranges between $0$ and $100$ or as $tip = percentage \times bill$
    if the percentage ranges between $0$ and $1$.

  \item Calculate the total bill as $total\_bill = bill+tip$

  \item Display to the user:
    \begin{itemize}
      \item The original bill (before the tip)
      \item The tip amount
      \item The total bill (including the tip)
    \end{itemize}

  \item Ask the user how many people will be splitting the total bill (remember
    to save the user's input into a variable so that it can be used later in the
    program and in the program's output).

  \item Calculate each person's share. For example, if you saved the number of
    people splitting the bill into a variable called {\em num\_people}, then you
    would calculate each person's share as $share =
    \frac{total\_bill}{num\_people}$. You should think (or rethink) your views
    concerning the data types you are using and whether a data conversion is
    required at this point in your program.

  \item Display to the user an exit message that is suitable for an academic
    setting.

\end{enumerate}

\noindent If a step says to ``ask the user'' this means the program should read
from a file. A sample run is:

\begin{verbatim}
  Gregory M. Kapfhammer Tue Sep 24 10:39:17 EDT 2020
  Timothy, welcome to the tip calculator.
  The amount of your bill is: $50.0
  The percentage you want to tip is: 15.0%
  Alright, that means your total tip will be: $7.5
  Timothy, your total bill is: $57.5
  Okay, I see that there are 2 people in your group.
  So, that means each person must pay: $28.75
  Thanks for using the tip calculator.
  Have an awesome day.
\end{verbatim}

With the exception of the name and date on which the program was run, your
\mainprogramsource{} should produce the same output as seen in the above
example. Additionally, you should try to edit the \mainprograminput{} file so
that it provides different inputs to the program. Does your \mainprogram{} still
produce the expected output? Remember, you need to run the program inside of a
Docker container, using a command like the following one. Don't forget that you
may have to adjust this command for your specific operating system and terminal
window, consulting the instructor if you need assistance with the use of Docker
desktop.

\begin{verbatim}
docker run -it --rm --name dockagator \
  -v "$(pwd):/project" \
  -v "$HOME/.dockagator:/root/.local/share" \
  gatoreducator/dockagator /bin/bash
\end{verbatim}

\marginnote{\discuss{Discuss input code}}[.35in] Remember, if you want to
\step{build} your program you can type the command \gradlebuild{} in your
terminal, thereby causing the Java compiler to check your program for errors and
get it ready to run. If you get any error messages, go back into the text editor
and try to figure out what you mis-typed and fix it. Once you have solved the
problem, make a note of the error and the solution for resolving it. Re-save
your program and then build it again by re-running the \gradlebuild{}. If you
cannot build \mainprogram{} correctly, then please talk with a technical leader
or the instructor. When you are coding don't forget to add comments explaining
the meaning and purpose of each line. Finally, make sure that you discuss the
provided source code, including the following fragment.

\begin{verbatim}
    try {
      tipInputsFile = new File("input/tip_inputs.txt");
      scanner = new Scanner(tipInputsFile);
    } catch (FileNotFoundException noFile) {
      System.out.println("Unable to locate file");
    }
\end{verbatim}

\marginnote{\think{Reflect on challenges}}[.7in] When all program errors are
eliminated, you can run your program by typing \gradlerun{} in the
terminal---this is the ``execute'' step that will run your program and produce
the designated output. You should see your name, today's date, and nine more
lines of text. Make sure there are spaces separating words in your output (did
you forget to put a space inside the quotation marks after your last name?). If
not, then repair the program and re-build and re-run it. Once the program runs,
please reflect on this process. What step did you find to be the most
challenging? Why? You should write your reflections in a file, called
\reflection{}, that uses the Markdown writing language. To complete this aspect
of the assignment, you should write high-quality paragraphs that report on your
experiences with the commands and Java code segments.

\section*{Checking the Correctness of Your Program and Writing}

\marginnote{\resource{Study style guides}}[.75in] As verified by the Checkstyle
tool, the source code for the \mainprogram{} file must adhere to all of the
requirements in the Google Java Style Guide available at
\url{https://google.github.io/styleguide/javaguide.html}. The Markdown file that
contains your reflection must adhere to the standards described in the Markdown
Syntax Guide \url{https://guides.github.com/features/mastering-markdown/}.
Finally, your \reflection{} file should adhere to the Markdown standards
established by the \step{Markdown linting} tool available at
\url{https://github.com/markdownlint/markdownlint/} and the writing standards
set by the \step{prose linting} tool from \url{http://proselint.com/}. Instead
of requiring you to manually check that your deliverables adhere to these
industry-accepted standards, the GatorGrader and Gradle tools that you will use
in this laboratory assignment makes it easy for you to automatically check if
your submission meets the standards for correctness.

Since this is not your first laboratory assignment, you will notice that the
provided source code does not contain all of the required imports at the top of
the Java source code file. This means that you will have to inspect the source
code from previous laboratory and practical assignments to review how to create
the imports in the \mainprogramsource{} file. Moreover, the provided source code
is missing many of the lines that are needed to pass the GatorGrader checks! For
instance, GatorGrader will check to ensure that the \mainprogram{} produces
exactly ten lines of output, that you use the \command{new Date()} construct in
the Java source code, and that the program correctly produces the same output as
seen on the previous page.

\marginnote{\caution{Write useful commits}}[.75in] To get started with the use
of GatorGrader, type the command \gatorgraderstart{} in your Docker container.
If your laboratory work does not meet all of the assignment's requirements, then
you will see a summary of the failing checks along with a statement giving the
percentage of checks that are currently passing. If you do have mistakes in your
assignment, then you will need to review GatorGrader's output, find the mistake,
and try to fix it. Once your program is building correctly, fulfilling at least
some of the assignment's requirements, you should transfer your files to GitHub
using the \gitcommit{} and \gitpush{} commands. For example, if you want to
signal that the \mainprogramsource{} file has been changed and is ready for
transfer to GitHub you would first type \gitcommitmainprogram{} in your
terminal, followed by typing \gitpush{} and checking to see that the transfer to
GitHub is successful. If you notice that transferring your code or writing to
GitHub did not work correctly, then read the errors in your terminal window and
please try to determine why, asking a technical leader or the course instructor
for assistance, if necessary.

After the course instructor enables \step{continuous integration} with a system
called Travis CI, when you use the \gitpush{} command to transfer your source
code to your GitHub repository, Travis CI will initialize a \step{build} of your
assignment, checking to see if it meets all of the requirements. If both your
source code and writing meet all of the established requirements, then you will
see a green \checkmark{} in the listing of commits in GitHub after awhile. If
your submission does not meet the requirements, a red \naughtmark{} will appear
instead. The instructor will reduce a student's grade for this assignment if the
red \naughtmark{} appears on the last commit in GitHub immediately before the
assignment's due date. Yet, if the green \checkmark{} appears on the last commit
in your GitHub repository, then you satisfied all of the main checks, thereby
allowing the course instructor to evaluate other aspects of your source code and
writing, as further described in the \step{Evaluation} section of this
assignment sheet. Unless you provide the instructor with documentation of the
extenuating circumstances that you are facing, no late work will be considered
towards your grade for this laboratory assignment.

In conclusion, here are some points to consider as you complete this laboratory
assignment:

\begin{enumerate}
  \setlength{\itemsep}{0pt}

\item The provided source code uses the {\tt Scanner} class; notice that it is
  imported at the top of the program. You may refer to the textbook for code
  demonstrating the use of the {\tt Scanner}; you need to use methods that
  support the input for variables of type {\tt String}, {\tt int}, and {\tt
  double}. Picking the wrong data types for your variables may cause the program
  to crash or to produce incorrect output. See the course instructor if you are
  not sure what data types to use.

\item Students who want to learn more about how to use the {\tt
  java.util.Scanner} class can refer to Section 2.6 of the textbook, focusing on
  the {\tt GasMileage} program discussed in class and provided in Listing 2.9.
  Make sure that you know how to construct and use a {\tt Scanner}.

\item You should think carefully about what data type you want to use for your
  variables.

\item As in past assignments, your program only needs to have one {\tt main}
  method.

\item To display a dollar sign, you can just type ``\$'' inside the string of
  your output statement.

\item Note that your program will alternate between printing and
  computing---this is okay!

\item Don't forget to review the assignment sheets from the previous laboratory
  and practical assignments as they contain insights that will support your
  completion of this assignment.

\end{enumerate}

\section*{Summary of the Required Deliverables}

\noindent Students do not need to submit printed source code or technical
writing for any assignment in this course. Instead, this assignment invites you
to submit, using GitHub, the following deliverables.

\begin{enumerate}

  \setlength{\itemsep}{0in}

\item Stored in \reflection{}, a multiple-paragraph reflection on the commands
  that you typed in the text editor and the terminal window. This Markdown-based
  document should explain the input, output, and behavior of each command and
  the challenges that you confronted when using it. For every challenge that you
  encountered, please explain your solution for it. You should also provide
  detailed answers to all the other technical questions.

\item A properly documented, well-formatted, and correct version of
  \mainprogramsource{} that both meets all of the established requirements and
  produces the desired output. In particular, please recall that this Java
  program should read inputs from a file, perform computations, and display the
  correct outputs in the terminal window.

\end{enumerate}

\section*{Evaluation of Your Laboratory Assignment}

Using a report that the instructor shares with you through the commit log in
GitHub, you will privately received a grade on this assignment and feedback on
your submitted deliverables. Your grade for the assignment will be a function of
the whether or not it was submitted in a timely fashion and if your program
received a green \checkmark{} indicating that it met all of the requirements.
Other factors will also influence your final grade on the assignment. In
addition to studying the efficiency and effectiveness and documentation of your
Java source code, the instructor will also evaluate the correctness of your
technical writing. If your submission receives a red \naughtmark{}, the
instructor will reduce your grade for the assignment. Finally, please remember
to read your GitHub repository's \program{README.md} file for a description of
the four grades that you will receive for this laboratory assignment.

\section*{Adhering to the Honor Code}

In adherence to the Honor Code, students should complete this assignment on an
individual basis. While it is appropriate for students in this class to have
high-level conversations about the assignment, it is necessary to distinguish
carefully between the student who discusses the principles underlying a problem
with others and the student who produces assignments that are identical to, or
merely variations on, someone else's work. Deliverables (e.g., Java source code
or Markdown-based technical writing) that are nearly identical to the work of
others will be taken as evidence of violating the \mbox{Honor Code}. Please see
the course instructor if you have questions about this policy.

\end{document}
